%-------------------------
% Resume in Latex
% Author : Puneet Gautam, puneet.gautam0206@gmail.com, punitrajgautam@uiit.ac.in
% License : MIT
%MIT License

%Copyright (c) 2023 Puneet Gautam

%Permission is hereby granted, free of charge, to any person obtaining a copy
%of this software and associated documentation files (the "Software"), to deal
%in the Software without restriction, including without limitation the rights
%to use, copy, modify, merge, publish, distribute, sublicense, and/or sell
%copies of the Software, and to permit persons to whom the Software is
%furnished to do so, subject to the following conditions:

%The above copyright notice and this permission notice shall be included in all
%copies or substantial portions of the Software.

%THE SOFTWARE IS PROVIDED "AS IS", WITHOUT WARRANTY OF ANY KIND, EXPRESS OR
%IMPLIED, INCLUDING BUT NOT LIMITED TO THE WARRANTIES OF MERCHANTABILITY,
%FITNESS FOR A PARTICULAR PURPOSE AND NONINFRINGEMENT. IN NO EVENT SHALL THE
%AUTHORS OR COPYRIGHT HOLDERS BE LIABLE FOR ANY CLAIM, DAMAGES OR OTHER
%LIABILITY, WHETHER IN AN ACTION OF CONTRACT, TORT OR OTHERWISE, ARISING FROM,
%OUT OF OR IN CONNECTION WITH THE SOFTWARE OR THE USE OR OTHER DEALINGS IN THE
%SOFTWARE.
%------------------------

%---- Required Packages and Functions ----

\documentclass[a4paper,11pt]{article}
\usepackage{latexsym}
\usepackage{xcolor}
\usepackage{float}
\usepackage{ragged2e}
\usepackage[empty]{fullpage}
\usepackage{wrapfig}
\usepackage{lipsum}
\usepackage{tabularx}
\usepackage{array}
\usepackage{color}
\usepackage{tikz}

\usepackage{titlesec}
\usepackage{geometry}
\usepackage{marvosym}
\usepackage{verbatim}
\usepackage{enumitem}
\usepackage[hidelinks]{hyperref}
\usepackage{fancyhdr}
\usepackage{multicol}
\usepackage{graphicx}
\usepackage{cfr-lm}
\usepackage[T1]{fontenc}
\setlength{\multicolsep}{0pt} 
\pagestyle{fancy}
\fancyhf{} % clear all header and footer fields
\fancyfoot{}
\renewcommand{\headrulewidth}{0pt}
\renewcommand{\footrulewidth}{0pt}
\geometry{left=0.6cm, top=0.5cm, right=0.6cm, bottom=0.5cm}
% Adjust margins
%\addtolength{\oddsidemargin}{-0.5in}
%\addtolength{\evensidemargin}{-0.5in}
%\addtolength{\textwidth}{1in}
\usepackage[most]{tcolorbox}
\tcbset{
	frame code={}
	center title,
	left=0pt,
	right=0pt,
	top=0pt,
	bottom=0pt,
	colback=gray!20,
	colframe=white,
	width=\dimexpr\textwidth\relax,
	enlarge left by=-2mm,
	boxsep=4pt,
	arc=0pt,outer arc=0pt,
}

\urlstyle{same}

\raggedright
\setlength{\tabcolsep}{0in}

% Sections formatting
\titleformat{\section}{
  \vspace{-4pt}\scshape\raggedright\large
}{}{0em}{}[\color{black}\titlerule \vspace{-7pt}]

%-------------------------
% Custom commands
\newcommand{\resumeItem}[2]{
  \item{
    \textbf{#1}{:\hspace{0.5mm}#2 \vspace{-0.5mm}}
  }
}

\newcommand{\resumePOR}[3]{
\vspace{0.5mm}\item
    \begin{tabular*}{0.97\textwidth}[t]{l@{\extracolsep{\fill}}r}
        \textbf{#1},\hspace{0.3mm}#2 & \textit{\small{#3}} 
    \end{tabular*}
    \vspace{-2mm}
}

\newcommand{\resumeSubheading}[4]{
\vspace{0.5mm}\item
    \begin{tabular*}{0.98\textwidth}[t]{l@{\extracolsep{\fill}}r}
        \textbf{#1} & \textit{\footnotesize{#4}} \\
        \textit{\footnotesize{#3}} &  \footnotesize{#2}\\
    \end{tabular*}
    \vspace{-2.4mm}
}

\newcommand{\resumeProject}[4]{
\vspace{0.5mm}\item
    \begin{tabular*}{0.98\textwidth}[t]{l@{\extracolsep{\fill}}r}
        \textbf{#1} & \textit{\footnotesize{#3}} \\
        \footnotesize{\textit{#2}} & \footnotesize{#4}
    \end{tabular*}
    \vspace{-2.4mm}
}

\newcommand{\resumeSubItem}[2]{\resumeItem{#1}{#2}\vspace{-4pt}}

% \renewcommand{\labelitemii}{$\circ$}
\renewcommand{\labelitemi}{$\vcenter{\hbox{\tiny$\bullet$}}$}

\newcommand{\resumeSubHeadingListStart}{\begin{itemize}[leftmargin=*,labelsep=0mm]}
\newcommand{\resumeHeadingSkillStart}{\begin{itemize}[leftmargin=*,itemsep=1.7mm, rightmargin=2ex]}
\newcommand{\resumeItemListStart}{\begin{justify}\begin{itemize}[leftmargin=3ex, rightmargin=2ex, noitemsep,labelsep=1.2mm,itemsep=0mm]\small}

\newcommand{\resumeSubHeadingListEnd}{\end{itemize}\vspace{2mm}}
\newcommand{\resumeHeadingSkillEnd}{\end{itemize}\vspace{-2mm}}
\newcommand{\resumeItemListEnd}{\end{itemize}\end{justify}\vspace{-2mm}}
\newcommand{\cvsection}[1]{%
\vspace{2mm}
\begin{tcolorbox}
    \textbf{\large #1}
\end{tcolorbox}
    \vspace{-4mm}
}

\newcolumntype{L}{>{\raggedright\arraybackslash}X}%
\newcolumntype{R}{>{\raggedleft\arraybackslash}X}%
\newcolumntype{C}{>{\centering\arraybackslash}X}%
%---- End of Packages and Functions ------

%-------------------------------------------
%%%%%%  CV STARTS HERE  %%%%%%%%%%%
%%%%%% DEFINE ELEMENTS HERE %%%%%%%
\newcommand{\name}{Kaimao Sheng} % Your Name
\newcommand{\course}{Objective: Robotic Engineer} % Your Course
\newcommand{\roll}{Your Roll No} % Your Roll No.
\newcommand{\phone}{1744247534} % Your Phone Number
\newcommand{\emaila}{kaimao.sheng@rwth-aachen.de} %Email 1
\newcommand{\emailb}{kaimao.sheng@rwth-aachen.de} %Email 2
\newcommand{\github}{https://github.com/KaiYakexi} %Github
\newcommand{\github}{https://github.com/KaiYakexi} %Github
\newcommand{\website}{https://www.linkedin.com/in/kaimao-sheng-806920176/} %Website
\newcommand{\linkedin}{https://www.linkedin.com/in/kaimao-sheng-806920176/} %linkedin
\newcommand\skill[2]{ 
    \begin{tikzpicture}
        \draw[fill=black!20,draw=none] (0,0) rectangle (10,0.3);
        \draw[fill=black!30,draw=none] (0,0) rectangle (#2,0.3);
        \node [above] at (0,0.3) {#1};
    \end{tikzpicture}
}

\usepackage{xcolor}



\begin{document}
\fontfamily{cmr}\selectfont
%----------HEADING-----------------
\parbox{2.35cm}{%

\includegraphics[width=2cm,clip]{kai.png}

}\parbox{\dimexpr\linewidth-2.8cm\relax}{
\begin{tabularx}{\linewidth}{L r}
  \textbf{\LARGE \name} & +49-\phone\\
  
  \course &  \href{mailto:\emailb}{\emailb}\\
   {Master Student at RWTH Aachen University} &  \href{https://github.com/KaiYakexi}{\color{blue}{GitHub}} \\ %$|$ \href{\website}{Website}\\
  {Working Student at Cerence GmbH} & \href{https://www.linkedin.com/in/kaimao-sheng-806920176/}{\color{blue}{LinkedIn}}
\end{tabularx}
}
\vspace{-2mm}
%-------------SUMMARY-----------------
\section{\textbf{Technical Skills}}

\textbf{Programming Skills}

\begin{tikzpicture}
\node [anchor=west] at (.1,.6) {Python};
\draw [fill=gray] (0,0) rectangle (5,.3);
\draw [fill={rgb:red,1;green,2;blue,3}] (0,0) rectangle (4.5,.3);
\end{tikzpicture}
\vspace{.05cm}
\begin{tikzpicture}
\node [anchor=west, font=\small] at (.1,.6) {C++};
\draw [fill=gray] (0,0) rectangle (5,.3);
\draw [fill={rgb:red,1;green,2;blue,3}] (0,0) rectangle (2.0,.3);
\end{tikzpicture}
\vspace{.05cm}
\begin{tikzpicture}
\node [anchor=west, font=\small] at (.1,.6) {Jupyter and its Extentions (eg. Jupyter-ros)};
\draw [fill=gray] (0,0) rectangle (5,.3);
\draw [fill={rgb:red,1;green,2;blue,3}] (0,0) rectangle (4.5,.3);
\end{tikzpicture}
\vspace{.05cm}
\begin{tikzpicture}
\node [anchor=west, font=\small] at (.1,.6) {Latex};
\draw [fill=gray] (0,0) rectangle (5,.3);
\draw [fill={rgb:red,1;green,2;blue,3}] (0,0) rectangle (4.5,.3);
\end{tikzpicture}
\vspace{.05cm}
\begin{tikzpicture}
\node [anchor=west, font=\small] at (.1,.6) {PowerShell, Bash, Batch Files};
\draw [fill=gray] (0,0) rectangle (5,.3);
\draw [fill={rgb:red,1;green,2;blue,3}] (0,0) rectangle (2.0,.3);
\end{tikzpicture}
\vspace{.05cm}
\begin{tikzpicture}
\node [anchor=west, font=\small] at (.1,.6) {Matlab (Simulation)};
\draw [fill=gray] (0,0) rectangle (5,.3);
\draw [fill={rgb:red,1;green,2;blue,3}] (0,0) rectangle (2.0,.3);
\end{tikzpicture}
\vspace{.2cm}


\textbf{Operating Systems}

\begin{tikzpicture}
\node [anchor=west, font=\small] at (.1,.6) {Ubuntu};
\draw [fill=gray] (0,0) rectangle (5,.3);
\draw [fill={rgb:red,1;green,2;blue,3}] (0,0) rectangle (4.5,.3);
\end{tikzpicture}
\vspace{.05cm}
\begin{tikzpicture}
\node [anchor=west, font=\small] at (.1,.6) {macOS};
\draw [fill=gray] (0,0) rectangle (5,.3);
\draw [fill={rgb:red,1;green,2;blue,3}] (0,0) rectangle (4.5,.3);
\end{tikzpicture}
\vspace{.05cm}
\begin{tikzpicture}
\node [anchor=west, font=\small] at (.1,.6) {Android};
\draw [fill=gray] (0,0) rectangle (5,.3);
\draw [fill={rgb:red,1;green,2;blue,3}] (0,0) rectangle (4.5,.3);
\end{tikzpicture}
\vspace{.05cm}
\begin{tikzpicture}
\node [anchor=west, font=\small] at (.1,.6) {Microsoft Windows};
\draw [fill=gray] (0,0) rectangle (5,.3);
\draw [fill={rgb:red,1;green,2;blue,3}] (0,0) rectangle (2.5,.3);
\end{tikzpicture}
\vspace{.2cm}

\textbf{Robotics}

\begin{tikzpicture}
\node [anchor=west, font=\small] at (.1,.6) {ML and CV Algorithms};
\draw [fill=gray] (0,0) rectangle (5,.3);
\draw [fill={rgb:red,1;green,2;blue,3}] (0,0) rectangle (4.0,.3);
\end{tikzpicture}
\vspace{.05cm}
\begin{tikzpicture}
\node [anchor=west, font=\small] at (.1,.6) {Robotic Operating Systems};
\draw [fill=gray] (0,0) rectangle (5,.3);
\draw [fill={rgb:red,1;green,2;blue,3}] (0,0) rectangle (3.5,.3);
\end{tikzpicture}
\begin{tikzpicture}
\node [anchor=west, font=\small] at (.1,.6) {Robotics Simulation and Modeling};
\draw [fill=gray] (0,0) rectangle (5,.3);
\draw [fill={rgb:red,1;green,2;blue,3}] (0,0) rectangle (3.0,.3);
\end{tikzpicture}
\vspace{.05cm}
\begin{tikzpicture}
\node [anchor=west, font=\small] at (.1,.6) {Control Systems Design};
\draw [fill=gray] (0,0) rectangle (5,.3);
\draw [fill={rgb:red,1;green,2;blue,3}] (0,0) rectangle (3.0,.3);
\end{tikzpicture}
%-----------EDUCATION-----------------
\section{\textbf{Education}}
\setlength{\tabcolsep}{5pt} % Default value: 6pt
% \renewcommand{\arraystretch}{1.1} % Default value: 1
\small{\begin{tabularx}
{\dimexpr\textwidth-2mm\relax}{|c|C|c|c|c|}
  \hline
  \textbf{Degree } & \textbf{Major in} & \textbf{University} & \textbf{GPA (German Standard)} & \textbf{Year}\\
  \hline
  M.Sc & Robotic Systems Engineering & RWTH Aachen University & 2,7 (Till 5th Sem) & 2020-Present\\
 
  \hline
  BEng & Mechanical Engineering & Shanghai Polytechnic University & 2,4 & 2016-2020 \\
  \hline
\end{tabularx}}
\vspace{-1mm}

%-----------EXPERIENCE-----------------
\section{\textbf{Experience}}
  \resumeSubHeadingListStart

    \resumeSubheading
      {Cerence GmbH }{Aachen, Germany}
      {Quality Assurance Software Engineer, Working Student}{October 2022 - Present}
      \resumeItemListStart
        \item {Working on the OneInfotainment project for Audi and Volkswagen. }
        \item {Review product requirements, user experience specifications, and technical design specifications for testability.}
        \item{Develop test strategy, test plan, and test case for the specific project.}
        \item{Conduct test plan/cases review with cross-functional teams to ensure the test coverage.}
        \item{Build and maintain requirement/test case/ test result traceability matrix.}
        \item{Enhance test automation framework for support of new features and improve test efficiency.}
        \item{Develop automation test tools in Python.}
        \item{Collaborate with multiple teams on different continents.}
        \item{Analyze test results along with failure analysis, utilizes fact/data-driven metrics to continuously improve test efficiency and effectiveness.}
        \item{Ensure measurable improvement of product quality by identifying, developing, and implementing innovative QA capabilities, strategies, approaches, and services.}
        \item{Escalate issues, concerns, and risk.}
      \resumeItemListEnd
      
    \resumeSubheading
      {Cybernetics Lab IMA of RWTH Aachen University }{Aachen, Germany}
      {{Student Researcher} - \href{https://github.com/KaiYakexi/U-net_training.git}{\color{blue}{IMA\_Github}}}{August 2022 - Present}
      \resumeItemListStart
        \item {Working on the research Project called: Clustering analysis in anomaly detection using In-situ monitoring geometry data during laser powder fusion bed manufacturing.}
        \item {Literature review in the related fields.}
        \item{Design testing parts using CAD/SOLIDWORKS for laser powder fusion bed.}
        \item{CT scanning to create a new layerwise image dataset at machine hall of WZL}
        \item{Defect analysis using myVGL and generate labels accordingly.}
        \item{Manul labelling using imageJ the original camera images during printing.}
        \item{Development of a U-net model for image segmentation.}
        \item{Supervised anomaly detection for varying geometry using in-situ monitoring data of laser powder fusion bed manufacturing processes. (TensorFlow using GPU, Jupyter notebook) }

      \resumeItemListEnd   

    \resumeSubheading
      {Institut für Kraftfahrzeuge of RWTH Aachen University  }{Aachen, Germany}
      {{Research Project} - \href{https://github.com/KaiYakexi/ACDC_project_ika_RWTH.git}{\color{blue}{IKA\_Github}}}{April 2022 - August 2022}
      \resumeItemListStart
        \item {Worked on a research Project called: Advanced filtering techniques during object tracking & fusion for self-driving cars. }
        \item {Literature review in the related fields.}
        \item{Research and implement an Extended Kalman filtering technique for object fusion and tracking in self-driving with special attention to nonlinear models and noise robustness. (created new ROS packages with C++ nodes)}
        \item{Quantitatively and qualitatively evaluate the performance of the tracking using the standard Kalman filter against the implementation used in the research.}
        \item{Using Jupyter to perform the research work.}

      \resumeItemListEnd 
    
    \resumeSubheading
      {Institut für Allgemeine Mechanik of RWTH Aachen University  }{Aachen, Germany}
      {{Internship} - \href{https://github.com/KaiYakexi/IMU_sensors.git}{\color{blue}{IAM\_Github}}}{November 2019 - April 2020}
      \resumeItemListStart
        \item {Collected and analyzed the experimental data from literature. }
        \item {Data cleaning for the sensors data of IMU using python. }
        \item{Supported the establishment of an efficient numerical approach for computational.}

      \resumeItemListEnd 

    \resumeSubheading
      {Institut für Allgemeine Mechanik of RWTH Aachen University  }{Aachen, Germany}
      {Bachelor Thesis}{November 2019 - April 2020}

      \resumeItemListStart
        \item {Optimization of IMU input data during gait analysis using neural network. (TensorFlow, Keras, Scikit-learn) }
        \item {Research paper score: 1,7 }

      \resumeItemListEnd 
%    \vspace{-1mm}
%    \resumeSubheading
%      {Company Name}{Location}
%      {Position}{June 2021 - July 2021}
%      \resumeItemListStart
%    \item {About work}
%    \item {About Work}
%    \resumeItemListEnd
    
    
    % \item {More work done } .....

    
  
      
  \resumeSubHeadingListEnd
\vspace{-6.5mm}
%-----------PROJECTS-----------------
\section{\textbf{Related Studies}}
\resumeSubHeadingListStart
    
    \resumeProject
      {Robotic Sensor Systems} %Project Name
      {WZL RWTH, Grade: 1,0} %Project Name, Location Name
      
      
      \resumeItemListStart
        \item {Learned different sensors used for various Robotics; 
Understood robots as metrological systems in industrial processes;
Learned current sensor technologies are presented and underlying physical fundamentals;
Learned the techniques of signal processing.}
        
      \resumeItemListEnd
    
     \vspace{-1mm}
     
    \resumeProject
      {Automated and Connected Driving Challenges} %Project Name
      {IKA RWTH, Grade: 1,7} %Project Name, Location Name
      
      
      \resumeItemListStart
        \item {Developed functions for automated and connected vehicles using Python and C++;
Integrated their developed functions into the Robot Operating System;
Trained Neural Networks for Computer Vision tasks with Tensorflow;
Learned how to use tools like Linux, Terminal, ROS, RVIZ, Juypter Notebooks, Git.}
        
      \resumeItemListEnd
     
    \resumeProject
      {Machine Learning} %Project Name
      {Lehrstuhl für Informatik, Passing rate: 13\% (out of 159 participants)} %Project Name, Location Name
      
      
      \resumeItemListStart
        \item {Performed K-Neighborhood density estimators, EM Algorithm
Least square linear classifiers, SVM, AdaBoost;
Trained A deeper network, Softmax-regression with backpropagation;
Trained Convolutional Neural Networks, Classification with CNN.}
        
      \resumeItemListEnd
    
      
    \resumeProject
      {Multibody Dynamics} %Project Name
      {IGMR RWTH, Grade: 2,3} %Project Name, Location Name
      
      
      \resumeItemListStart
        \item {Studied theoretical foundations of multibody dynamics, including kinematics, dynamics, constraints, joints, and numerical integration methods;
Gained practical experience with MATLAB/Simulink software to simulate complex mechanical systems with multiple moving parts;
Explored applications of multibody dynamics in vehicle dynamics, robotics, and biomechanics;
Learned about advanced topics in multibody dynamics, including flexible bodies, contact mechanics, and optimization.}
        
      \resumeItemListEnd
    
      
    \resumeProject
      {Reinforcement Learning and Learning-based Control} %Project Name
      {Chair of Data Science in Mechanical Engineering: Ongoing} %Project Name, Location Name
      
      
      \resumeItemListStart
        \item {Studied reinforcement learning algorithms and their application in control systems;
Learned about model-based and model-free reinforcement learning techniques;
Examined Q-learning, policy gradients, and actor-critic methods;
Explored deep reinforcement learning and its applications;
Studied learning-based control methods and their role in robotics and automation;
Analyzed the performance of learning-based controllers through simulations and experiments;
Implemented reinforcement learning algorithms on real-world control problems;
Gained experience with Python, TensorFlow, and OpenAI Gym in the context of reinforcement learning.}
        
      \resumeItemListEnd
    
      
    \resumeProject
      {Mechatronic and Control for Production Plants} %Project Name
      {WZL RWTH} %Project Name, Location Name
      
      
      \resumeItemListStart
        \item {Learned the principles of mechatronics and their application in industrial automation;
Acquired practical skills in the design and implementation of control systems for production plants;
Explored the integration of mechanical, electrical, and software components in production systems;
Studied advanced control strategies for optimizing production efficiency and quality;
Analyzed case studies of mechatronic systems in real-world manufacturing scenarios.}
        
      \resumeItemListEnd
    
      
    \resumeProject
      {Other Related Studies at RWTH} %Project Name
      {M.Sc Robotic Systems Engineering} %Project Name, Location Name
      


      \resumeItemListStart
        \item {Computer Vision}
        \item {Simulation of Robotic Systems, Sensors and Environment}
        \item {Artificial Intelligence and Data Analytics for Engineers}
        \item {Advanced Robotic Kinematics and Dynamics}
        \item {Robotic Systems}
        \item {Control Engineering}
      \resumeItemListEnd    
     
    
   

\vspace{-1.5mm}





\end{document}

%-------------------------------------------
\vfill
\center{\footnotesize Last updated: \today}

\end{document}